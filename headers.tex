% -*- root:main.tex -*-

\usepackage{xspace,amsmath,amsfonts,amssymb,hyperref,tikz,multirow,wrapfig,tabularx,hhline,enumitem,stmaryrd, placeins  }

\usepackage{siunitx}

\newcommand{\F}{\ensuremath{\mathcal{F}}\xspace}
\newcommand{\adv}{\ensuremath{\mathcal{A}}\xspace}

%%%%%%%%%%%%%%%%%%%%

\newcommand{\etal}{{\sl et~al.~}\xspace}
\newcommand{\eg}{{\sl e.g.}\xspace}
\newcommand{\ie}{{\sl i.e.}\xspace}
\newcommand{\apriori}{{\sl a~priori\/}\xspace}
\newcommand{\f}[1]{\ensuremath{\mathcal{F}_{\sf #1}}\xspace}

% F style encoding
%\newcommand{\encode}[3]{\ensuremath{F^\textsf{#1}_{#2}(#3)}}
% bracket style encoding
\newcommand{\encode}[3][]{\ensuremath{\llbracket #3\rrbracket^\textsf{#1}_{#2}}}


%%%%%%%%%%%%%%%%%%%%%%

\newcommand{\algo}[1]{\ensuremath{\text{\sf #1}}\xspace}
\newcommand{\command}[1]{\ensuremath{\text{\sc #1}}\xspace}

%%%%%%%%%%%%%%%%%%%%

%\iffullversion
    % \usepackage{amsthm}
    % \newtheorem{theorem}{Theorem}
    % \newtheorem{definition}[theorem]{Definition}
    % \newtheorem{claim}[theorem]{Claim}
    % \newtheorem{lemma}[theorem]{Lemma}
%\else
%    \newtheorem{theorem}{Theorem}
%\fi

\newtheorem{corol}[theorem]{Corollary}
\newtheorem{assumption}[theorem]{Assumption}
\newtheorem{obs}[theorem]{Observation}
\newtheorem{conj}[theorem]{Conjecture}

\newenvironment{proofof}[1]{\begin{proof}[Proof of #1.]}{\end{proof}}
\newenvironment{proofsketch}{\begin{proof}[Proof Sketch]}{\end{proof}}

%%%%%%%%%%%%%%%%%%%%%%%%%

\newcommand{\namedref}[2]{\hyperref[#2]{#1~\ref*{#2}}}
%% if you don't like it, use this instead:
%\newcommand{\namedref}[2]{#1~\ref{#2}}
\newcommand{\chapterref}[1]{\namedref{Chapter}{#1}}
\newcommand{\sectionref}[1]{\namedref{Section}{#1}}
\newcommand{\theoremref}[1]{\namedref{Theorem}{#1}}
\newcommand{\definitionref}[1]{\namedref{Definition}{#1}}
\newcommand{\corollaryref}[1]{\namedref{Corollary}{#1}}
\newcommand{\obsref}[1]{\namedref{Observation}{#1}}
\newcommand{\lemmaref}[1]{\namedref{Lemma}{#1}}
\newcommand{\claimref}[1]{\namedref{Claim}{#1}}
\newcommand{\figureref}[1]{\namedref{Figure}{#1}}
\newcommand{\hybridref}[1]{\namedref{Hybrid}{#1}}
\newcommand{\stepref}[1]{\namedref{Step}{#1}}
\newcommand{\subfigureref}[2]{\hyperref[#1]{Figure~\ref*{#1}#2}}
\newcommand{\equationref}[1]{\namedref{Equation}{#1}}
\newcommand{\appendixref}[1]{\namedref{Appendix}{#1}}
\newcommand{\sid}{\textsf{sid}}

\definecolor{darkred}{rgb}{0.5, 0, 0} 
\definecolor{darkgreen}{rgb}{0, 0.5, 0} 
\definecolor{darkblue}{rgb}{0,0,0.5} 

\hypersetup{
    colorlinks=true,
    linkcolor=darkred,
    citecolor=darkgreen,
    urlcolor=darkblue   
}

% normal mathcal
\DeclareMathAlphabet{\mathcal}{OMS}{cmsy}{m}{n}

%%%%%%%%%%

\newcommand{\todo}[1]{%
%    \mbox{}% prevent marginpar from being on previous paragraph
%    \marginpar{%
%        \colorbox{red!80!black}{\textcolor{white}{to-do}}%
%        \vspace*{-22pt}% hack!
%    }%
    \textcolor{red}{#1}%
}

%%%%%%%%%%%%%%%%%%%%%%%%
\newlist{hybrids}{enumerate}{3}
\setlist[hybrids]{label=\textit{Hybrid \arabic* },labelwidth=10pt,align=left,leftmargin=*,itemindent=38.3pt,topsep=6pt,itemsep=4pt, listparindent=10pt,parsep=2pt}
\setlist[hybrids,1]{start=0} 



% Beginning of definitions
%\usepackage{environ}
%
%\newif\ifappendix
%\NewEnviron{maybeappendix}[1]
%{\ifappendix
%	\expandafter\global\expandafter\let\csname putmaybeappendix#1\endcsname\BODY%
%	\else
%	\expandafter\newcommand\csname putmaybeappendix#1\endcsname{}\BODY%
%	\fi}
%\newcommand{\putmaybeappendix}[1]{\csname putmaybeappendix#1\endcsname}
% End of definitions

%Added from FJNT16 paper
\newcommand{\A}{\textsf{A}\xspace} 
\newcommand{\B}{\textsf{B}\xspace} 
\newcommand{\C}{\textsf{C}\xspace} 
\newcommand{\E}{\textsf{E}\xspace} 
\newcommand{\Pone}{\ensuremath{\mathsf{P_1}}\xspace}
\newcommand{\Ptwo}{\ensuremath{\mathsf{P_2}}\xspace}
\newcommand{\mG}{\ensuremath{\mathcal{G}}\xspace}
\newcommand{\mA}{\ensuremath{\mathcal{A}}\xspace}
\newcommand{\mS}{\ensuremath{\mathcal{S}}\xspace}
\newcommand{\mT}{\ensuremath{\mathcal{T}}\xspace}
\newcommand{\mB}{\ensuremath{\mathcal{B}}\xspace}
\newcommand{\mP}{\ensuremath{\mathcal{P}}\xspace}
\newcommand{\mZ}{\ensuremath{\mathcal{Z}}\xspace}
\newcommand{\snd}{\ensuremath{P_s}\xspace}
\newcommand{\rec}{\ensuremath{P_r}\xspace}

\renewcommand{\Game}{\operatorname{Game}}
\newcommand{\Ind}{\operatorname{Ind}}
\newcommand{\prop}{\operatorname{prop}}
\newcommand{\gamePRG}{\operatorname{PRG}}
\newcommand{\PRG}{\ensuremath{\textsf{PRG}}\xspace}
\newcommand{\prg}{\ensuremath{\text{prg}}\xspace}
\newcommand{\Det}{\operatorname{Det}}
\newcommand{\Der}{\operatorname{Der}}

\newcommand{\Ex}{\ensuremath{\mathsf{Ex}}\xspace}
\newcommand{\Guess}{\textsc{Guess}}

% Sets
\newcommand{\N}{\mathbb{N}}
\newcommand{\Z}{\mathbb{Z}}
\newcommand{\R}{\mathbb{R}}
%%\newcommand{\K}{\mathbb{K}}
\newcommand{\Field}{\mathbb{F}}
\newcommand{\Q}{\mathbb{Q}}
\newcommand{\zo}{\ensuremath{\{0,1\}}}
\newcommand{\zp}{\ensuremath{\mathbb{Z}_p}}
\newcommand{\zn}{\ensuremath{\mathbb{Z}_n}}
\newcommand{\zq}{\ensuremath{\mathbb{Z}_q}}
\newcommand{\zpstar}{\ensuremath{\mathbb{Z}_p^*}}
\newcommand{\znstar}{\ensuremath{\mathbb{Z}_n^*}}
\newcommand{\zqstar}{\ensuremath{\mathbb{Z}_q^*}}

% Macros
\newcommand{\cmd}[1]{\ensuremath{\texttt{#1}}}
\newcommand{\nin}{\ensuremath{\not \in}\xspace}
\newcommand{\myand}{\ensuremath{\wedge}\xspace}
\newcommand{\myor}{\ensuremath{\vee}\xspace}
\newcommand{\myset}[1]{\ensuremath{\left\{ #1 \right\}}\xspace}
\newcommand{\myoset}[1]{\ensuremath{\left( #1 \right)}\xspace}
\newcommand{\myosset}[1]{\ensuremath{( #1 )}\xspace}
\newcommand{\ord}[1]{\ensuremath{\left | #1 \right |}\xspace}
\newcommand{\myabs}[1]{\ensuremath{\left | #1 \right |}\xspace}
\newcommand{\mysetminus}[1]{\ensuremath{\setminus { #1 }}\xspace}
\newcommand{\WLOG}{\emph{w.l.o.g.}\xspace}
\newcommand{\mymat}[2]{\ensuremath{\text{Mat}_{#1 \times #2}}\xspace}
\newcommand{\xor}{\ensuremath{\oplus}}
\newcommand\concat{\mathbin{\|}}
\newcommand{\ant}[1]{\ensuremath{\mathsf{#1}}}
\newcommand{\key}[3][K]{\ensuremath{{#1}_{#3}^{#2}}\xspace}
\newcommand{\negl}{\operatorname{negl}}
\newcommand{\ham}{\operatorname{ham}}
\newcommand{\notsym}{\ensuremath{\mathds{1}}\xspace}
\newcommand{\bigoh}{\ensuremath{\mathcal{O}}\xspace}

\newcommand{\sample}{\ensuremath{\in_R}\xspace}
\newcommand{\store}{\ensuremath{\leftarrow}\xspace}
\newcommand{\true}{\cmd{true}\xspace}
\newcommand{\false}{\cmd{false}\xspace}
\newcommand{\bool}{\cmd{bool}\xspace}
\newcommand{\ind}[1][]{\ensuremath{\stackrel{#1}{\approx}}\xspace}

% Variables
\newcommand{\csec}{\ensuremath{\kappa}\xspace}
\newcommand{\ssec}{\ensuremath{s}\xspace}
\newcommand{\codesize}{\ensuremath{n}\xspace}
\newcommand{\codedistance}{\ensuremath{d}\xspace}
\newcommand{\codeword}{\ensuremath{\vec{c}}\xspace}
\newcommand{\code}{\ensuremath{\mathcal{C}}\xspace}
\newcommand{\len}{\ensuremath{k}\xspace}

% Functionalities
\newcommand{\ifunc}{\ensuremath{\mathcal{F}}\xspace}
\newcommand{\HCOM}{\ensuremath{\ifunc_{\textsf{HCOM}}}\xspace}
\newcommand{\EHCOM}{\ensuremath{\ifunc_{\textsf{EHCOM}}}\xspace}
\newcommand{\OT}{\ensuremath{\ifunc_{\textsf{OT}}}\xspace}
\newcommand{\ROT}{\ensuremath{\ifunc_{\textsf{ROT}}}\xspace}

% Protocol Specific
\newcommand{\PHCOM}{\ensuremath{\Uppi_{\textsf{HCOM}}}\xspace}
\newcommand{\PEHCOM}{\ensuremath{\Uppi_{\textsf{EHCOM}}}\xspace}

\newcommand{\Init}{\textbf{Init}\xspace}
\newcommand{\Commit}{\textbf{Commit}\xspace}
\newcommand{\ChosenCommit}{\textbf{Chosen-Commit}\xspace}
\newcommand{\ExtendedOpen}{\textbf{Extended-Open}\xspace}
\newcommand{\Open}{\textbf{Open}\xspace}
\newcommand{\Addition}{\textbf{Addition}\xspace}
\newcommand{\BatchOpen}{\textbf{Batch-Open}\xspace}

\newcommand{\commitcounter}{\ensuremath{\mathcal{T}}\xspace}
\newcommand{\currentcommit}{\ensuremath{\mathcal{J}}\xspace}
\newcommand{\ComVer}{\ensuremath{\mathsf{ComVer}}\xspace}
\newcommand{\numofcommits}{\ensuremath{\gamma}\xspace}
\newcommand{\numoftotalcommits}{\ensuremath{\delta}\xspace}
\newcommand{\numberoflcc}{\fsec}


% Functionality commands
\newcommand{\init}{\cmd{init}\xspace}
\newcommand{\open}{\cmd{open}\xspace}
\newcommand{\opened}{\cmd{opened}\xspace}
\newcommand{\opening}{\cmd{opening}\xspace}
\newcommand{\batchopen}{\cmd{batch-open}\xspace}
\newcommand{\batchopened}{\cmd{batch-opened}\xspace}
\newcommand{\extendedopen}{\cmd{extended-open}\xspace}
\newcommand{\extendedopened}{\cmd{extended-opened}\xspace}
\newcommand{\commit}{\cmd{commit}\xspace}
\newcommand{\chosencommit}{\cmd{chosen-commit}\xspace}
\newcommand{\committed}{\cmd{committed}\xspace}
\newcommand{\chosencommitted}{\cmd{chosen-committed}\xspace}
\newcommand{\corruptcommit}{\cmd{corrupt-commit}\xspace}
\newcommand{\nocorrupt}{\cmd{no-corrupt}\xspace}
\newcommand{\success}{\cmd{success}\xspace}
\newcommand{\add}{\cmd{add}\xspace}
\newcommand{\receipt}{\cmd{receipt}\xspace}
\newcommand{\random}{\cmd{random}\xspace}
\newcommand{\chosen}{\cmd{chosen}\xspace}
\newcommand{\chosenreceipt}{\cmd{chosen-receipt}\xspace}
\newcommand{\sender}{\cmd{sender}\xspace}
\newcommand{\transfer}{\cmd{transfer}\xspace}
\newcommand{\linear}{\cmd{linear}\xspace}
\newcommand{\corruptsender}{\cmd{corrupt-sender}\xspace}
\newcommand{\corruptreceiver}{\cmd{corrupt-receiver}\xspace}
\newcommand{\receiver}{\cmd{receiver}\xspace}
\newcommand{\deliver}{\cmd{deliver}\xspace}
\newcommand{\abort}{\cmd{abort}\xspace}
\newcommand{\ssid}{\cmd{ssid}\xspace}
\newcommand{\cid}{\cmd{cid}\xspace}
\newcommand{\otid}{\cmd{otid}\xspace}

\newcommand{\cmsg}{\ensuremath{\vec{m}}\xspace}
\newcommand{\rmsg}{\ensuremath{\vec{r}}\xspace}
\newcommand{\nmsg}{\ensuremath{\vec{n}}\xspace}

% Reference commands
\newcommand{\figref}[1]{Fig.~\ref{fig:#1}}
\newcommand{\figlab}[1]{\label{fig:#1}}
\newcommand{\thmref}[1]{Theorem~\ref{thm:#1}}
\newcommand{\thmlab}[1]{\label{thm:#1}}
\newcommand{\lemref}[1]{Lemma~\ref{lem:#1}}
\newcommand{\lemlab}[1]{\label{lem:#1}}
\newcommand{\colref}[1]{Corollary~\ref{col:#1}}
\newcommand{\collab}[1]{\label{col:#1}}
\newcommand{\secref}[1]{Section~\ref{sec:#1}}
\newcommand{\seclab}[1]{\label{sec:#1}}
\newcommand{\applab}[1]{\label{app:#1}}
\newcommand{\appref}[1]{Appendix~\ref{app:#1}}
\newcommand{\chapref}[1]{Chapter~\ref{chap:#1}}
\newcommand{\chaplab}[1]{\label{chap:#1}}
\newcommand{\tablab}[1]{\label{tab:#1}}
\newcommand{\tabref}[1]{Table~\ref{tab:#1}}
\newcommand{\defref}[1]{Definition~\ref{def:#1}}
\newcommand{\deflab}[1]{\label{def:#1}}
\newcommand{\gameref}[1]{Game~\ref{game:#1}}
\newcommand{\gamelab}[1]{\label{game:#1}}
\newcommand{\eqlab}[1]{\label{eq:#1}}
\newcommand{\eqnref}[1]{(\ref{eq:#1})}

\newcommand{\CC}{C\&C\xspace} %cut-and-choose
                              
\newcommand{\RO}{\textsf{RO}\xspace}