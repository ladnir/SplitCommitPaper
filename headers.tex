\usepackage{xspace,amsmath,amsfonts,amssymb,hyperref,tikz,multirow,wrapfig,tabularx,hhline,enumitem,stmaryrd, placeins  }

\usepackage{siunitx}

\newcommand{\F}{\ensuremath{\mathcal{F}}\xspace}
\newcommand{\adv}{\ensuremath{\mathcal{A}}\xspace}

%%%%%%%%%%%%%%%%%%%%

\newcommand{\etal}{{\sl et~al.~}}
\newcommand{\eg}{{\sl e.g.}}
\newcommand{\ie}{{\sl i.e.}}
\newcommand{\apriori}{{\sl a~priori\/}\xspace}
\newcommand{\f}[1]{\ensuremath{\mathcal{F}_{\sf #1}}}

% F style encoding
%\newcommand{\encode}[3]{\ensuremath{F^\textsf{#1}_{#2}(#3)}}
% bracket style encoding
\newcommand{\encode}[3][]{\ensuremath{\llbracket #3\rrbracket^\textsf{#1}_{#2}}}


%%%%%%%%%%%%%%%%%%%%%%

\newcommand{\algo}[1]{\ensuremath{\text{\sf #1}}\xspace}
\newcommand{\command}[1]{\ensuremath{\text{\sc #1}}\xspace}

%%%%%%%%%%%%%%%%%%%%

%\iffullversion
    \usepackage{amsthm}
    \newtheorem{theorem}{Theorem}
    \newtheorem{definition}[theorem]{Definition}
    \newtheorem{claim}[theorem]{Claim}
    \newtheorem{lemma}[theorem]{Lemma}
%\else
%    \newtheorem{theorem}{Theorem}
%\fi

\newtheorem{corol}[theorem]{Corollary}
\newtheorem{assumption}[theorem]{Assumption}
\newtheorem{obs}[theorem]{Observation}
\newtheorem{conj}[theorem]{Conjecture}

\newenvironment{proofof}[1]{\begin{proof}[Proof of #1.]}{\end{proof}}
\newenvironment{proofsketch}{\begin{proof}[Proof Sketch]}{\end{proof}}

%%%%%%%%%%%%%%%%%%%%%%%%%

\newcommand{\namedref}[2]{\hyperref[#2]{#1~\ref*{#2}}}
%% if you don't like it, use this instead:
%\newcommand{\namedref}[2]{#1~\ref{#2}}
\newcommand{\chapterref}[1]{\namedref{Chapter}{#1}}
\newcommand{\sectionref}[1]{\namedref{Section}{#1}}
\newcommand{\theoremref}[1]{\namedref{Theorem}{#1}}
\newcommand{\definitionref}[1]{\namedref{Definition}{#1}}
\newcommand{\corollaryref}[1]{\namedref{Corollary}{#1}}
\newcommand{\obsref}[1]{\namedref{Observation}{#1}}
\newcommand{\lemmaref}[1]{\namedref{Lemma}{#1}}
\newcommand{\claimref}[1]{\namedref{Claim}{#1}}
\newcommand{\figureref}[1]{\namedref{Figure}{#1}}
\newcommand{\hybridref}[1]{\namedref{Hybrid}{#1}}
\newcommand{\stepref}[1]{\namedref{Step}{#1}}
\newcommand{\subfigureref}[2]{\hyperref[#1]{Figure~\ref*{#1}#2}}
\newcommand{\equationref}[1]{\namedref{Equation}{#1}}
\newcommand{\appendixref}[1]{\namedref{Appendix}{#1}}
\newcommand{\sid}{\textsf{sid}}

\definecolor{darkred}{rgb}{0.5, 0, 0} 
\definecolor{darkgreen}{rgb}{0, 0.5, 0} 
\definecolor{darkblue}{rgb}{0,0,0.5} 

\hypersetup{
    colorlinks=true,
    linkcolor=darkred,
    citecolor=darkgreen,
    urlcolor=darkblue   
}

% normal mathcal
\DeclareMathAlphabet{\mathcal}{OMS}{cmsy}{m}{n}

%%%%%%%%%%

\newcommand{\todo}[1]{%
%    \mbox{}% prevent marginpar from being on previous paragraph
%    \marginpar{%
%        \colorbox{red!80!black}{\textcolor{white}{to-do}}%
%        \vspace*{-22pt}% hack!
%    }%
    \textcolor{red}{#1}%
}

%%%%%%%%%%%%%%%%%%%%%%%%
\newlist{hybrids}{enumerate}{3}
\setlist[hybrids]{label=\textit{Hybrid \arabic* },labelwidth=10pt,align=left,leftmargin=*,itemindent=38.3pt,topsep=6pt,itemsep=4pt, listparindent=10pt,parsep=2pt}
\setlist[hybrids,1]{start=0} 



% Beginning of definitions
%\usepackage{environ}
%
%\newif\ifappendix
%\NewEnviron{maybeappendix}[1]
%{\ifappendix
%	\expandafter\global\expandafter\let\csname putmaybeappendix#1\endcsname\BODY%
%	\else
%	\expandafter\newcommand\csname putmaybeappendix#1\endcsname{}\BODY%
%	\fi}
%\newcommand{\putmaybeappendix}[1]{\csname putmaybeappendix#1\endcsname}
% End of definitions
