%\documentclass[11pt,letterpaper]{article}


    \documentclass{article}
    \usepackage[letterpaper,margin=1in]{geometry}

% -*- root:main.tex -*-

\usepackage{xspace,amsmath,amsfonts,amssymb,hyperref,tikz,multirow,wrapfig,tabularx,hhline,enumitem,stmaryrd, placeins  }

\usepackage{siunitx}

\newcommand{\F}{\ensuremath{\mathcal{F}}\xspace}
\newcommand{\adv}{\ensuremath{\mathcal{A}}\xspace}

%%%%%%%%%%%%%%%%%%%%

\newcommand{\etal}{{\sl et~al.~}}
\newcommand{\eg}{{\sl e.g.}}
\newcommand{\ie}{{\sl i.e.}}
\newcommand{\apriori}{{\sl a~priori\/}\xspace}
\newcommand{\f}[1]{\ensuremath{\mathcal{F}_{\sf #1}}}

% F style encoding
%\newcommand{\encode}[3]{\ensuremath{F^\textsf{#1}_{#2}(#3)}}
% bracket style encoding
\newcommand{\encode}[3][]{\ensuremath{\llbracket #3\rrbracket^\textsf{#1}_{#2}}}


%%%%%%%%%%%%%%%%%%%%%%

\newcommand{\algo}[1]{\ensuremath{\text{\sf #1}}\xspace}
\newcommand{\command}[1]{\ensuremath{\text{\sc #1}}\xspace}

%%%%%%%%%%%%%%%%%%%%

%\iffullversion
    \usepackage{amsthm}
    \newtheorem{theorem}{Theorem}
    \newtheorem{definition}[theorem]{Definition}
    \newtheorem{claim}[theorem]{Claim}
    \newtheorem{lemma}[theorem]{Lemma}
%\else
%    \newtheorem{theorem}{Theorem}
%\fi

\newtheorem{corol}[theorem]{Corollary}
\newtheorem{assumption}[theorem]{Assumption}
\newtheorem{obs}[theorem]{Observation}
\newtheorem{conj}[theorem]{Conjecture}

\newenvironment{proofof}[1]{\begin{proof}[Proof of #1.]}{\end{proof}}
\newenvironment{proofsketch}{\begin{proof}[Proof Sketch]}{\end{proof}}

%%%%%%%%%%%%%%%%%%%%%%%%%

\newcommand{\namedref}[2]{\hyperref[#2]{#1~\ref*{#2}}}
%% if you don't like it, use this instead:
%\newcommand{\namedref}[2]{#1~\ref{#2}}
\newcommand{\chapterref}[1]{\namedref{Chapter}{#1}}
\newcommand{\sectionref}[1]{\namedref{Section}{#1}}
\newcommand{\theoremref}[1]{\namedref{Theorem}{#1}}
\newcommand{\definitionref}[1]{\namedref{Definition}{#1}}
\newcommand{\corollaryref}[1]{\namedref{Corollary}{#1}}
\newcommand{\obsref}[1]{\namedref{Observation}{#1}}
\newcommand{\lemmaref}[1]{\namedref{Lemma}{#1}}
\newcommand{\claimref}[1]{\namedref{Claim}{#1}}
\newcommand{\figureref}[1]{\namedref{Figure}{#1}}
\newcommand{\hybridref}[1]{\namedref{Hybrid}{#1}}
\newcommand{\stepref}[1]{\namedref{Step}{#1}}
\newcommand{\subfigureref}[2]{\hyperref[#1]{Figure~\ref*{#1}#2}}
\newcommand{\equationref}[1]{\namedref{Equation}{#1}}
\newcommand{\appendixref}[1]{\namedref{Appendix}{#1}}
\newcommand{\sid}{\textsf{sid}}

\definecolor{darkred}{rgb}{0.5, 0, 0} 
\definecolor{darkgreen}{rgb}{0, 0.5, 0} 
\definecolor{darkblue}{rgb}{0,0,0.5} 

\hypersetup{
    colorlinks=true,
    linkcolor=darkred,
    citecolor=darkgreen,
    urlcolor=darkblue   
}

% normal mathcal
\DeclareMathAlphabet{\mathcal}{OMS}{cmsy}{m}{n}

%%%%%%%%%%

\newcommand{\todo}[1]{%
%    \mbox{}% prevent marginpar from being on previous paragraph
%    \marginpar{%
%        \colorbox{red!80!black}{\textcolor{white}{to-do}}%
%        \vspace*{-22pt}% hack!
%    }%
    \textcolor{red}{#1}%
}

%%%%%%%%%%%%%%%%%%%%%%%%
\newlist{hybrids}{enumerate}{3}
\setlist[hybrids]{label=\textit{Hybrid \arabic* },labelwidth=10pt,align=left,leftmargin=*,itemindent=38.3pt,topsep=6pt,itemsep=4pt, listparindent=10pt,parsep=2pt}
\setlist[hybrids,1]{start=0} 



% Beginning of definitions
%\usepackage{environ}
%
%\newif\ifappendix
%\NewEnviron{maybeappendix}[1]
%{\ifappendix
%	\expandafter\global\expandafter\let\csname putmaybeappendix#1\endcsname\BODY%
%	\else
%	\expandafter\newcommand\csname putmaybeappendix#1\endcsname{}\BODY%
%	\fi}
%\newcommand{\putmaybeappendix}[1]{\csname putmaybeappendix#1\endcsname}
% End of definitions

%Added from FJNT16 paper
\newcommand{\A}{\textsf{A}\xspace} 
\newcommand{\B}{\textsf{B}\xspace} 
\newcommand{\C}{\textsf{C}\xspace} 
\newcommand{\E}{\textsf{E}\xspace} 
\newcommand{\Pone}{\ensuremath{\mathsf{P_1}}\xspace}
\newcommand{\Ptwo}{\ensuremath{\mathsf{P_2}}\xspace}
\newcommand{\mG}{\ensuremath{\mathcal{G}}\xspace}
\newcommand{\mA}{\ensuremath{\mathcal{A}}\xspace}
\newcommand{\mS}{\ensuremath{\mathcal{S}}\xspace}
\newcommand{\mT}{\ensuremath{\mathcal{T}}\xspace}
\newcommand{\mB}{\ensuremath{\mathcal{B}}\xspace}
\newcommand{\mP}{\ensuremath{\mathcal{P}}\xspace}
\newcommand{\mZ}{\ensuremath{\mathcal{Z}}\xspace}
\newcommand{\snd}{\ensuremath{P_s}\xspace}
\newcommand{\rec}{\ensuremath{P_r}\xspace}

\renewcommand{\Game}{\operatorname{Game}}
\newcommand{\Ind}{\operatorname{Ind}}
\newcommand{\prop}{\operatorname{prop}}
\newcommand{\gamePRG}{\operatorname{PRG}}
\newcommand{\PRG}{\ensuremath{\textsf{PRG}}\xspace}
\newcommand{\prg}{\ensuremath{\text{prg}}\xspace}
\newcommand{\Det}{\operatorname{Det}}
\newcommand{\Der}{\operatorname{Der}}

\newcommand{\Ex}{\ensuremath{\mathsf{Ex}}\xspace}
\newcommand{\Guess}{\textsc{Guess}}

% Sets
\newcommand{\N}{\mathbb{N}}
\newcommand{\Z}{\mathbb{Z}}
\newcommand{\R}{\mathbb{R}}
%%\newcommand{\K}{\mathbb{K}}
\newcommand{\Field}{\mathbb{F}}
\newcommand{\Q}{\mathbb{Q}}
\newcommand{\zo}{\ensuremath{\{0,1\}}}
\newcommand{\zp}{\ensuremath{\mathbb{Z}_p}}
\newcommand{\zn}{\ensuremath{\mathbb{Z}_n}}
\newcommand{\zq}{\ensuremath{\mathbb{Z}_q}}
\newcommand{\zpstar}{\ensuremath{\mathbb{Z}_p^*}}
\newcommand{\znstar}{\ensuremath{\mathbb{Z}_n^*}}
\newcommand{\zqstar}{\ensuremath{\mathbb{Z}_q^*}}

% Macros
\newcommand{\cmd}[1]{\ensuremath{\texttt{#1}}}
\newcommand{\nin}{\ensuremath{\not \in}\xspace}
\newcommand{\myand}{\ensuremath{\wedge}\xspace}
\newcommand{\myor}{\ensuremath{\vee}\xspace}
\newcommand{\myset}[1]{\ensuremath{\left\{ #1 \right\}}\xspace}
\newcommand{\myoset}[1]{\ensuremath{\left( #1 \right)}\xspace}
\newcommand{\myosset}[1]{\ensuremath{( #1 )}\xspace}
\newcommand{\ord}[1]{\ensuremath{\left | #1 \right |}\xspace}
\newcommand{\myabs}[1]{\ensuremath{\left | #1 \right |}\xspace}
\newcommand{\mysetminus}[1]{\ensuremath{\setminus { #1 }}\xspace}
\newcommand{\WLOG}{\emph{w.l.o.g.}\xspace}
\newcommand{\mymat}[2]{\ensuremath{\text{Mat}_{#1 \times #2}}\xspace}
\newcommand{\xor}{\ensuremath{\oplus}}
\newcommand\concat{\mathbin{\|}}
\newcommand{\ant}[1]{\ensuremath{\mathsf{#1}}}
\newcommand{\key}[3][K]{\ensuremath{{#1}_{#3}^{#2}}\xspace}
\newcommand{\negl}{\operatorname{negl}}
\newcommand{\ham}{\operatorname{ham}}
\newcommand{\notsym}{\ensuremath{\mathds{1}}\xspace}
\newcommand{\bigoh}{\ensuremath{\mathcal{O}}\xspace}

\newcommand{\sample}{\ensuremath{\in_R}\xspace}
\newcommand{\store}{\ensuremath{\leftarrow}\xspace}
\newcommand{\true}{\cmd{true}\xspace}
\newcommand{\false}{\cmd{false}\xspace}
\newcommand{\bool}{\cmd{bool}\xspace}
\newcommand{\ind}[1][]{\ensuremath{\stackrel{#1}{\approx}}\xspace}

% Variables
\newcommand{\csec}{\ensuremath{\kappa}\xspace}
\newcommand{\ssec}{\ensuremath{s}\xspace}
\newcommand{\codesize}{\ensuremath{n}\xspace}
\newcommand{\codedistance}{\ensuremath{d}\xspace}
\newcommand{\codeword}{\ensuremath{\vec{c}}\xspace}
\newcommand{\code}{\ensuremath{\mathcal{C}}\xspace}
\newcommand{\len}{\ensuremath{k}\xspace}

% Functionalities
\newcommand{\ifunc}{\ensuremath{\mathcal{F}}\xspace}
\newcommand{\HCOM}{\ensuremath{\ifunc_{\textsf{HCOM}}}\xspace}
\newcommand{\EHCOM}{\ensuremath{\ifunc_{\textsf{EHCOM}}}\xspace}
\newcommand{\OT}{\ensuremath{\ifunc_{\textsf{OT}}}\xspace}
\newcommand{\ROT}{\ensuremath{\ifunc_{\textsf{ROT}}}\xspace}

% Protocol Specific
\newcommand{\PHCOM}{\ensuremath{\Uppi_{\textsf{HCOM}}}\xspace}
\newcommand{\PEHCOM}{\ensuremath{\Uppi_{\textsf{EHCOM}}}\xspace}

\newcommand{\Init}{\textbf{Init}\xspace}
\newcommand{\Commit}{\textbf{Commit}\xspace}
\newcommand{\ChosenCommit}{\textbf{Chosen-Commit}\xspace}
\newcommand{\ExtendedOpen}{\textbf{Extended-Open}\xspace}
\newcommand{\Open}{\textbf{Open}\xspace}
\newcommand{\Addition}{\textbf{Addition}\xspace}
\newcommand{\BatchOpen}{\textbf{Batch-Open}\xspace}

\newcommand{\commitcounter}{\ensuremath{\mathcal{T}}\xspace}
\newcommand{\currentcommit}{\ensuremath{\mathcal{J}}\xspace}
\newcommand{\ComVer}{\ensuremath{\mathsf{ComVer}}\xspace}
\newcommand{\numofcommits}{\ensuremath{\gamma}\xspace}
\newcommand{\numoftotalcommits}{\ensuremath{\delta}\xspace}
\newcommand{\numberoflcc}{\fsec}


% Functionality commands
\newcommand{\init}{\cmd{init}\xspace}
\newcommand{\open}{\cmd{open}\xspace}
\newcommand{\opened}{\cmd{opened}\xspace}
\newcommand{\opening}{\cmd{opening}\xspace}
\newcommand{\batchopen}{\cmd{batch-open}\xspace}
\newcommand{\batchopened}{\cmd{batch-opened}\xspace}
\newcommand{\extendedopen}{\cmd{extended-open}\xspace}
\newcommand{\extendedopened}{\cmd{extended-opened}\xspace}
\newcommand{\commit}{\cmd{commit}\xspace}
\newcommand{\chosencommit}{\cmd{chosen-commit}\xspace}
\newcommand{\committed}{\cmd{committed}\xspace}
\newcommand{\chosencommitted}{\cmd{chosen-committed}\xspace}
\newcommand{\corruptcommit}{\cmd{corrupt-commit}\xspace}
\newcommand{\nocorrupt}{\cmd{no-corrupt}\xspace}
\newcommand{\success}{\cmd{success}\xspace}
\newcommand{\add}{\cmd{add}\xspace}
\newcommand{\receipt}{\cmd{receipt}\xspace}
\newcommand{\random}{\cmd{random}\xspace}
\newcommand{\chosen}{\cmd{chosen}\xspace}
\newcommand{\chosenreceipt}{\cmd{chosen-receipt}\xspace}
\newcommand{\sender}{\cmd{sender}\xspace}
\newcommand{\transfer}{\cmd{transfer}\xspace}
\newcommand{\linear}{\cmd{linear}\xspace}
\newcommand{\corruptsender}{\cmd{corrupt-sender}\xspace}
\newcommand{\corruptreceiver}{\cmd{corrupt-receiver}\xspace}
\newcommand{\receiver}{\cmd{receiver}\xspace}
\newcommand{\deliver}{\cmd{deliver}\xspace}
\newcommand{\abort}{\cmd{abort}\xspace}
\newcommand{\ssid}{\cmd{ssid}\xspace}
\newcommand{\cid}{\cmd{cid}\xspace}
\newcommand{\otid}{\cmd{otid}\xspace}

\newcommand{\cmsg}{\ensuremath{\vec{m}}\xspace}
\newcommand{\rmsg}{\ensuremath{\vec{r}}\xspace}
\newcommand{\nmsg}{\ensuremath{\vec{n}}\xspace}

% Reference commands
\newcommand{\figref}[1]{Fig.~\ref{fig:#1}}
\newcommand{\figlab}[1]{\label{fig:#1}}
\newcommand{\thmref}[1]{Theorem~\ref{thm:#1}}
\newcommand{\thmlab}[1]{\label{thm:#1}}
\newcommand{\lemref}[1]{Lemma~\ref{lem:#1}}
\newcommand{\lemlab}[1]{\label{lem:#1}}
\newcommand{\colref}[1]{Corollary~\ref{col:#1}}
\newcommand{\collab}[1]{\label{col:#1}}
\newcommand{\secref}[1]{Section~\ref{sec:#1}}
\newcommand{\seclab}[1]{\label{sec:#1}}
\newcommand{\applab}[1]{\label{app:#1}}
\newcommand{\appref}[1]{Appendix~\ref{app:#1}}
\newcommand{\chapref}[1]{Chapter~\ref{chap:#1}}
\newcommand{\chaplab}[1]{\label{chap:#1}}
\newcommand{\tablab}[1]{\label{tab:#1}}
\newcommand{\tabref}[1]{Table~\ref{tab:#1}}
\newcommand{\defref}[1]{Definition~\ref{def:#1}}
\newcommand{\deflab}[1]{\label{def:#1}}
\newcommand{\gameref}[1]{Game~\ref{game:#1}}
\newcommand{\gamelab}[1]{\label{game:#1}}
\newcommand{\eqlab}[1]{\label{eq:#1}}
\newcommand{\eqnref}[1]{(\ref{eq:#1})}

\newcommand{\CC}{C\&C\xspace} %cut-and-choose
                              
\newcommand{\RO}{\textsf{RO}\xspace}
    \renewcommand{\floatpagefraction}{.99}
    \renewcommand{\topfraction}{.99}
    \renewcommand{\textfraction}{.01}
    
\usepackage{listings}
\usepackage{xcolor}
\definecolor{listinggray}{gray}{0.9}
\definecolor{lbcolor}{rgb}{0.9,0.9,0.9}
\lstset{
	backgroundcolor=\color{lbcolor},
	tabsize=4,    
	%   rulecolor=,
	language=[GNU]C++,
	basicstyle=\scriptsize,
	upquote=true,
	aboveskip={1.5\baselineskip},
	columns=fixed,
	showstringspaces=false,
	extendedchars=false,
	breaklines=true,
	prebreak = \raisebox{0ex}[0ex][0ex]{\ensuremath{\hookleftarrow}},
	frame=single,
	numbers=left,
	showtabs=false,
	showspaces=false,
	showstringspaces=false,
	identifierstyle=\ttfamily,
	keywordstyle=\color[rgb]{0,0,1},
	commentstyle=\color[rgb]{0.026,0.112,0.095},
	stringstyle=\color[rgb]{0.627,0.126,0.941},
	numberstyle=\color[rgb]{0.205, 0.142, 0.73},
	%        \lstdefinestyle{C++}{language=C++,style=numbers}’.
}
\lstset{
	backgroundcolor=\color{lbcolor},
	tabsize=4,
	language=C++,
	captionpos=b,
	tabsize=3,
	frame=lines,
	numbers=left,
	numberstyle=\tiny,
	numbersep=5pt,
	breaklines=true,
	showstringspaces=false,
	basicstyle=\footnotesize,
	%  identifierstyle=\color{magenta},
	keywordstyle=\color[rgb]{0,0,1},
	commentstyle=\color{Darkgreen},
	stringstyle=\color{red}
}

\newcommand*\samefootnote[1][\value{footnote}]{\footnotemark[#1]}

\begin{document}
	
	\title{SplitCommit: Implementing Homomorphic UC Commitments}
	
	
	\author{
		Peter Rindal\thanks{Oregon State University,  {\sf rindalp@oregonstate.edu}. This author is supported by an ARCS foundation fellowship.}, Roberto  Trifiletti\thanks{Aarhus University, {\sf roberto@cs.au.dk}.}
		}
		
		\maketitle

	
	\begin{abstract}
	In this white paper we present SplitCommit, a portable and efficient C++ implementation of the additively homomorphic commmitment scheme of Frederiksen, Jakobsen, Nielsen, and Trifiletti \cite{DBLP:conf/tcc/FrederiksenJNT16}. This document is meant to serve as a light-weight introduction to the scheme and give the reader an overview of its applicability for any given scenario. We include implementation design discussions and performance numbers as well as general usage guidelines. Finally we also include a small tutorial on how to use the library.

	\end{abstract}
	
	
	\pagestyle{plain}
	
	% -*- root:main.tex -*-

\section{Introduction}

A cryptographic commitment scheme can be thought of as the digital equivalent to a  physical box with an associated lock. Imagine a setting where a sending party puts some message inside the box, locks it, and then sends it to a receiving party. As an immediate observation, since the box is locked the receiver cannot figure out what is inside by mere possession of the box. At some later point in time though, the sender can send the key that unlocks the box, and since the box has been in the receiver's custody the whole time it is guaranteed that the message inside is indeed the original message. Translating the above scenario to the digital world, we call the first property \textit{hiding} (receiver can't look inside the box) and the second \textit{binding} (the sender cannot change its content) and it is these properties that are realized by a cryptographic commitment scheme. We use the terms `committing to a message' as the equivalent to sending the locked box and `decomitting to a message' to mean sending the key that unlocks it.

In this white paper we present SplitCommit, an efficient and portable C++14 implementation of the recent additively homomorphic commitment scheme of \cite{DBLP:conf/tcc/FrederiksenJNT16}. In short, a commitment scheme is additively homomorphic if it allows the sending party to decommit to the sum of previously committed messages. This enhanced ability turns out not only to be useful in several applications, but also imply very low communication overhead when decommitting to a batch of messages.

\subsection{Overview of \cite{DBLP:conf/tcc/FrederiksenJNT16}}
The commitment scheme of \cite{DBLP:conf/tcc/FrederiksenJNT16} provide UC-secure additively homomorphic two-party commitments using 1-out-of-2 Oblivious Transfer (OT) as the main building block. The scheme supports committing to vectors of length $k$ over any finite field $\Field$, but the SplitCommit implementation put forward in this work only implements commitments for the binary field $\Field_2 = \myset{0,1}$ and for $k \in \myset{1,128}$. In other words the source code works solely for committing to bit-strings of length 1 or 128. However, other bit-string lengths could be added in a straight forward way. Due to the binary field restriction we sometimes refer to the scheme as merely XOR-homomorphic. In addition to OT, the scheme also makes use of a linear error correction code (ECC) with parameters $[n,k,d]_\Field$ where $n$ is the codeword length, $k$ is the message length, and $d$ is the minimum distance for which we require $d \geq \ssec$ for statistical security parameter $s$. As provided, the implementation only includes ECC parameters for the widely used case of $s=40$. 

The \cite{DBLP:conf/tcc/FrederiksenJNT16} scheme is tailored for committing to a batch of $m$ messages and intuitively works in the following way. The sender picks $m$ random messages ($k$-bit strings) and encode these using the above-mentioned ECC into $n$-bit strings. It then creates a 2-additive secret sharing of each of the resulting codewords. We view the two parts of the secret shares as making up two bit matrices, where the $i$'th secret sharing is located in the $i$'th column of the two matrices. For each $m$ bit \emph{row} of the two matrices, the parties perform an OT to transfer one of the two rows. As a result, for each of the $n$ bit positions in the secret shares, the receiver learns the corresponding bit value of either the first or second secret share, and it learns the same share-position for all $m$ codewords. Intuitively, by allowing the receiver to learn parts of each of the secret sharings, where the part that was learned is unknown to the sender, the probability for the sender to successfully change the value that it is committed to without being caught is negligible. Moreover, the receiver learns nothing about that committed value due to the receiver only learning disjoint parts of the secret sharing.

In order to later decommit to a committed value the sender sends both shares of the corresponding codeword to the receiver which can then check for each position that the value reported matches the local entry-share for this value. If they all match, and the reconstructed message is a valid codeword it accepts the message, else it aborts. In order to decommit to the XOR of messages, the sender instead sends the XOR of the corresponding shares and the receiver verifies using the XOR of its local shares. This works due to the ECC and additive secret sharing both being linear operations and therefore their composition is linear as well. 

For security to hold, the scheme requires one additional step where sender gives an argument that it did not cheat. In particular a critical consistency check needs to be applied in order to ensure the sender sends secret shares that actually adds up to valid codewords. Also, in order to implement the above approach more efficiently, the parties run the 1-out-of-2 OTs on short $\csec$-bit strings and expand these using a pseudorandom generator (PRG), as opposed to running 1-out-of-2 OTs on the potentially much longer $m$-bit strings which would be much less efficient.
	
	% -*- root:main.tex -*-

\section{Applications and Usage Guidelines}
In this section we highlight some of the works in the cryptographic literature that benefit directly from an efficient XOR-homomorphic commitment scheme implementation.

The LEGO paradigm for maliciously secure two-party computation (2PC) using garbled circuits was proposed in \cite{DBLP:conf/tcc/NielsenO09} as an alternative to the whole-circuit Cut-and-Choose (\CC) approach of \cite{DBLP:conf/eurocrypt/LindellP07} and many following works. The central idea of whole-circuit \CC is to achieve malicious security by constructing many garbled circuit and checking some fraction of them, where each circuit computes the same target function $f$. With this approach one can guarantee security except with probability $2^{-\ssec}$ by garbling and sending $O(\ssec)$ circuits. With LEGO, these checks are instead performed on individually garbled NAND gates, which are later combined in a secure way so that they compute $f$. Via a combinatorial argument it can be shown that in then suffices to send $O(\frac{\ssec}{\log(|f|)})$ garbled circuits, a substantial asymptotic savings over whole-circuit \CC. However, in order to combine the constructed NAND gates so that they securely compute $f$, the gate constructor is required to commit to all wires of all garbled gates using XOR-homomorphic commitments. This is needed as it enables secure ``soldering'' of an output wire of one garbled gate onto the input wire of another, thus building a garbled circuit using the NAND gates as building blocks. Although the LEGO paradigm is asymptotically more efficient, the overhead of three commitments per NAND gate is substantial when looking at concrete performance and required communication. As the amount of commitments grows linearly in the size of the final circuit $f$ it is evident that efficient constructions and implementations of XOR-homomorphic commitments are instrumental for the efficiency of LEGO-like 2PC protocols. Recently, the work of \cite{NST17} implemented the TinyLEGO protocol of \cite{DBLP:journals/iacr/FrederiksenJNT15} using an early version of what is now the SplitCommit library. In fact, the SplitCommit project is based on a fork of \cite{NST17}, which has since adopted it as an external library.

In addition to the above 2PC application, XOR-homomorphic commitments have also been proposed as building blocks for evaluating RAM programs with malicious security using garbled circuits \cite{DBLP:conf/eurocrypt/AfsharHMR15} and achieving non-interactive maliciously secure 2PC in the batched setting \cite{DBLP:journals/iacr/MohasselR17}.
	% -*- root:main.tex -*-

\subsection{Usage Guidelines}
\seclab{usage_guidelines}
It is a known fact that any commitment scheme in the standard model (non random oracle~\cite{DBLP:conf/ccs/BellareR93}) which is UC-secure \cite{DBLP:conf/focs/Canetti01} requires communicating at least the message size, both in the commitment and decommitment phase. The work of \cite{DBLP:conf/eurocrypt/GarayIKW14} was the first protocol to meet these bounds asymptotically for the non-homomorphic case, whereas \cite{DBLP:conf/tcc/FrederiksenJNT16} achieved the same for additively homomorphic commitments. It is an interesting observation, that although \cite{DBLP:conf/tcc/FrederiksenJNT16} does not achieve the above when decommitting to a single value, one can exploit the homomorphic property of the scheme to achieve when considering many or large messages. The technique is called batch opening and it works as follows for decommitting to $n$ messages:
\begin{enumerate}
  \item The sender sends $n$ messages directly to the receiver, which he claims are the decommitted values.
  \item The receiver then sends a challenge to the sender specifying $\ssec$ random linear combinations of all $n$ messages.
  \item The sender computes these $\ssec$ combinations, using the homomorphic property of the scheme, and decommits toward the receiver.
  \item The receiver verifies the decommits and in addition computes the same combinations on the $n$ initially claimed messages. If these match the correct decommitted combinations, he is guaranteed that the claimed messages are in fact what the sender originally committed to.
\end{enumerate}

We highlight that the above technique also works for a single large message, as the sender is this case can commit to the message block-wise to obtain the same effect.

\paragraph{Trade-offs and Considerations.}
We advocate that even applications where the \textit{homomorphic} property is not required, SplitCommit might be a right fit nonetheless. This is largely due to the concrete efficiency of the scheme as demonstrated in \secref{performance} and by the above batch decommit optimization. In addition, when committing to random messages, the scheme of \cite{DBLP:conf/tcc/FrederiksenJNT16} (and SplitCommit) allows for sub-linear communication in the commitment phase as these can be defined from the outputs of the PRGs directly. We here list the most profound pros and cons to consider before using the scheme:
\begin{description}
  \item[Pros:]
  ~
  \begin{enumerate}
     \item Concretely efficient, fraction of a microsecond to commit/decommit on standard hardware (see \secref{performance}).
     \item Additive homomorphism allowing for more flexibility and applicability.
     \item Possibility of sub-linear communication when committing to random messages.
     \item Standard-model assumptions, in particular no random oracle is required.
     \item If in a context already requiring OTs (such as general 2PC), the seed OTs required for the commitment scheme can be realized using OT extension, thus greatly minimizing the initial setup cost.
   \end{enumerate}
   \item[Cons:]
   ~
   \begin{enumerate}
     \item Two-party, directional commitments only. If the receiver needs to commit to messages as well, a separate instance of the scheme needs to be setup in the other direction.
     \item Requires OT to bootstrap. Depending on setting this can be too expensive to setup, especially when only a few commitments to small messages are required.
     \item Committing is interactive (single round).\footnote{For some settings, the Fiat-Shamir optimization~\cite{DBLP:conf/crypto/FiatS86} in the random oracle model can be applied to eliminate this interaction.}
     \item Non-interactive decommitment (non-batch) has $\sim4\times$ communication overhead for 128-bit messages with $\ssec=40$.\samefootnote
   \end{enumerate}
\end{description}



\subsection{Comparison to Random Oracle commitments}
It is widely know that a random oracle (RO) $h:\zo^* \to \zo^l$ can be used to implement a commitment scheme in a very straight-forward way. In order to commit to a message $m$, the sender computes $c = h(m\|r)$ for some fresh random string $r$ and sends $c$ to the receiver. To decommit, the sender reveals $m$ and $r$ to the receiver, who can then check whether $c = h(m\|r)$ or not. The above scheme can be proven secure in UC-framework if $h$ is modeled as an extractable, programmable\footnote{Programmability is required to make the commitment equivocable.} random oracle model. With this assumption the simulator in the proof of security has complete control over the hash function and can both see what values are hashed (extract) and decide on a particular output (program). This is clearly a very strong assumption on any real-world function (and impossible to achieve in general \cite{DBLP:conf/stoc/CanettiGH98}), and we note for completeness that several other commitment schemes exists using weaker flavors of ROs such as~\cite{DBLP:conf/tcc/HofheinzM04,DBLP:conf/ccs/Canetti0S14}. The latter schemes are however less efficient than the above sketched out scheme, and as such we take a worst-case approach by comparing SplitCommit to the most efficient RO based construction. Even so, it can clearly be seen in by the following discussion that the performance of SplitCommit is on par with or supersede that of the RO scheme for several interesting sets of parameters.

\begin{description}
  \item[Starting Assumption:] The RO scheme requires the existence of an extractable, programmable random oracle to be provably UC-secure, whereas SplitCommit requires the existence of UC OT. Although UC OT is also a strong starting requirement it is strictly weaker than that of a random oracle.
  \item[Communication:] Typically the random oracle $h$ would be instantiated using a cryptographic hash function, say SHA-256 which then means the output size is 256 bits. Therefore no matter the size of the message (or if it's random or not), the sender needs to send 256 bits per committed value. This can be both a strength and a weakness, depending on the message size. Clearly if the message is large, the RO scheme can achieve sub-linear communication when committing, even for chosen messages. However for smaller messages, such as \eg commitments to 128-bit strings, committing to a value has a 2x overhead. For decommitting, the sender needs to send the messages and the random string $r$ which is typically around 128 bits long. Again for the case of 128-bit messages yields a 2x overhead.

  For SplitCommit the cost of committing can be sublinear for random messages and essentially the message size for large chosen messages. The exact trade-offs depends on the security parameter and ECC parameters, but as an example, for 128-bit messages and $\ssec=40$ it requires 134 bits to commit to a random message and 128-bits to decommit using batch decommit.

  \item[Computation:] From the performance section we see that SplitCommit can commit and decommit to a value in less than a microsecond (when considering many values). Based on the observations of \cite{DBLP:conf/pkc/CascudoDDGNT15} this roughly equivalent to a SHA-256 evaluation.
  \item[Homomorphism:] SplitCommit allows additively homomorphic decommits whereas the RO scheme does not.
  \item[Setup Cost:] SplitCommit requires an intial setup phase involving running a number of potentially expensive seed OTs. However, depending on the scope and complexity of the application as a whole this can be considered anything from a complete deal-breaker to a negligible added cost. The RO scheme requires no such setup.
  \item[Interactiveness:] The RO scheme is non-interactive in both the commitment and decommitment phase, whereas SplitCommit in it's most efficient mode is interactive in both phases. However as previously mentioned, using the Fiat-Shamir heuristic (which as well requires a RO assumption) this can be turned non-interactive as well.
\end{description}


\begin{figure}
	\small
	\begin{tabular}{|c||c|c|c|c|c|c|c|}
		\hline
		Scheme                 & Setup & Homomorphic & Interactive &\shortstack{Random \\Commit} &  Commit & Decommit & \shortstack{Batch \\Decommit} \\ \hline
		\cite{DBLP:conf/tcc/FrederiksenJNT16} &  $\kappa$ OTs  &     \textbf{Yes}     &     Yes     &   \textbf{134 bits}    &   262 bits    & 524 bits &    \textbf{128 bits}    \\ \hline
		Random Oracle             &      \textbf{ No}       &     No      &     \textbf{No}      &   256 bits    &   \textbf{256 bits}    & \textbf{256 bits} &    256 bits    \\\hline
	\end{tabular}
	\caption{Comparison to the standard random oracle commitment scheme.}
\end{figure}

	% -*- root:main.tex -*-

\section{Implementing}

To obtain a high performance implementations, several of the most heavily used operations had to be significantly optimized over the naive methods. In this section we high light the most important ones. The task which is most costly in this protocol is the matrix transpose that maps $m$ secret sharings of $n$ bit codewords to its transpose consisting of $n$ entries, each $m$ bits long. In order to perform this operation efficiently, employ optimizations originally developed for OT extension protocols. The main technique is to used special instructions available on modern Intel CPUs to accelerate the computation. The most notable of these instruction is \texttt{\_mm\_movemask\_epi8} which takes 16 bytes as input and returns a 16 bit number consisting of the least significant bit of each byte. By performing 8 of these operations interleaved with shifting the 16 bytes to the right allows for the efficient transpose of a $8\times 16$ bit matrix, requiring roughly $20\times$ few instructions. This optimization was first observed in the cryptography community by the authors of the OT extension protocol~\cite{DBLP:conf/crypto/KellerOS15} and later described in a related work~\cite{DBLP:conf/ccs/KellerOS16}. While implementing this work and the related libOTe~\cite{libOTe} project, subsequently improved on this observation have been made. In particular, we employ Intel vector operations to allow 8  simultaneously\footnote{Most modern Intel CPUs support 8 vector lanes.} \texttt{\_mm\_movemask\_epi8} instructions to be performed, resulting in even greater throughput. In total these optimization represent several orders of magnitude improvement over a naive implementation and were essential in obtaining high performance.


Another computationally intensive operation is the construction of codewords that are subsequently secret shared. The most straight forward way to construct codewords over the $\mathbb{F}_2^n$ is by multiplying the the plaintext word interpreted as a bit vector by a bit matrix known as the generator. Due to the use of a linear code, the result of this multiplication will be the desired codeword. One important observation made by \cite{DBLP:conf/tcc/FrederiksenJNT16} is that and linear code can be placed in systematic form where codewords are the input concatenated with the a series of parity bits. As such, only the parity bits need be computing using matrix multiplication. However, this computation can still dominate the running time if implemented naively. Namely, the naive method is to test each bit of the input and then add the corresponding row of the generator matrix to a running sum if the bit is set. However, when performing this operations millions or more times, it can require significant time primarily due to the CPU not being able to effectively doing branch predication on the ``random" input bits, resulting in very poor instruction pipelining. 

Intuitively, we overcome this challenge by performing this conditional operation on a whole byte as opposed to single bits. This has several advantages: 1) fewer conditionals are performed. 2) branch prediction is not much more effective due to being conditioned on a whole byte.  3) The resulting operations can compatible with instruction vectorization. To enable this optimization, a pre-processing on the generator matrix itself is performed where for each set of 8 rows of the matrix, all 256 possible additive combinations are constructed and stored. The original matrix multiplication can then be reduced to using the bytes on the input to index into the corresponding of this expended generator matrix. To then achieve optimal performance, this indexing operation can be vectorized, allowing 8 indexing operations to be performed in parallel. 

Our final optimization which is widely known in the cryptography community is the use of AES-NI instruction set for performing fast pseudo-random number generation (PRNG). The implementation as is directly supports commitments on random values. These values are obtained by first performing $n$ so called base OTs on random and short strings. We then use these these short strings to generate the the two matrices of secret shared codeword. In particular, the base OTs are used as seeds to a PRNG. In all modern intel CPUs, the AES-NI instruction set allows for fast encryption via special hardware support. However, by placing AES in counter-mode and interpreting it as a PRNG. As with the previous optimizations, it is critical for performance that these AES operations be vectorized, there at least 8 blocks of output are computed in parallel.
%In this section, we can share the insight that were gained from implementing it. This includes the SSE instructions for fast trnaspose. Your trick for doing the consistency check with mul128 instructions, and my fast linear code optimization.
	\section{Performance}

To demonstrate the scalability of the implementation, benchmarks have been performed in an Intel Xeon server consisting of 2 36-cores Intel Xeon CPU E5-2699 v3 @ 2.30GHz and 256GB of RAM. In the single threaded setting on a 10Gbps LAN, $n=2^{24}$ commitments on random 128 bit values required 11 seconds, resulting in 0.65 microseconds per commitment. When decommitting via sending the two secret shares for all $n$ commitments, the implementation requires 5.4 seconds, 0.32 microseconds per decommitment. The same operation performed using batch decommit, the implementation requires 4.6 seconds, 0.27 microseconds per decommitment. Furthermore, homomorphic operations can be performed extremely fast, simply requiring  a few XOR instructions per operations, requiring less than 1 seconds for $n$ operations. 

In addition, we benchmark our implementation in the 100Mbps WAN setting with a 80ms round trip latency. For $n=2^{24}$, committing  required 75 seconds, 4.5 microseconds per commitment. Decommiting via sending both secret shares required 505 seconds while the more communication efficient batch decommitment method required 316 seconds, 18.9 microseconds per decommitment.



When performing bit commitments in the LAN setting, $n=2^{24}$ commitments to random bits required 1.3 seconds and decommitting required 2 seconds. Interestingly, batch decommitting is slightly slower, requiring 2.3 seconds to decommit. We attribute this to the reduction in communication being out weighed by the add computation that is required. Indeed, in the 100Mbps WAN setting batch decommitting is faster, requiring 60 seconds as opposed to 61.

In terms of communication overheads, committing to a random 128 bit value requires is 134 bits per commitment, with an added 128 overhead for chosen message. Decommitting requires 524 bits per while batch decommitting only requires 128 bits per with a constant additive overhead of 20960 bits. Committing to bits requires 40 bits of communication and decommitting requires 80 bits. Batch decommitting only requires a single bit of communication per with a constant additive overhead of 3200 bits.
	
	% -*- root:main.tex -*-

\section{Building SplitCommit}

SplitCommit and its dependencies can be found at \url{https://github.com/AarhusCrypto/SplitCommit}. This project utilizes the cmake build systems and capable of integrating into larger build systems with minimal overhead. Apart from cmake and other basic utilities, the implementation builds on the libOTe~\cite{libOTe} library which requires the c++ utility library Boost~\cite{boost} and cryptography libary Miracl~\cite{miracl}. For complete details on how to build these utilities, we refer the reader to the readme file contained in the git repository.


	% -*- root:main.tex -*-

\section{Tutorial}
SplitCommit and its dependencies can be found at \url{https://github.com/AarhusCrypto/SplitCommit}. This project utilizes the cmake build systems and capable of integrating into larger build systems with minimal overhead. Apart from cmake and other basic utilities, the implementation builds on the libOTe~\cite{libOTe} library which requires the C++ utility library Boost~\cite{boost} and cryptography libary Miracl~\cite{miracl}. For complete details on how to build these utilities, we refer the reader to the readme file contained in the git repository.



%
%Giving read code examples here could be good. I was thinking we could show how to set of networking, and a single two thread program that commits to 100 values or something.
%
%We should also show how to perform homomorphic operations on them.
%
%Then we can add batch decommit to the mix.
%
%Finally, show how to handle chosen message with homomorphic operations on them.
%
%---------------------------------------------------------------------------------------
%

\subsection{Random Commitments}
We now turn to giving a basic overview of how to use the SplitCommit library by providing minimal code snippets. First, let us consider the case of committing to \texttt{num\_commits} random commitments. For a given bit size, the primary class can be constructed as
\begin{lstlisting}
#include "split-commit/split-commit-rec.h"
#include "split-commit/split-commit-snd.h"
...
SplitCommitSender commit_snd(input_bit_size);
SplitCommitReceiver commit_rec(input_bit_size);
\end{lstlisting}

The commitments themselves are held in a vector like class called \texttt{BYTEArrayVector}. Due to the nature of the commitment, the sender must have two such containers, one to hold each share of the commitment. The receiver will then hold some combination of these two shares in their own \texttt{BYTEArrayVector}.
\begin{lstlisting}
std::array<BYTEArrayVector, 2> send_commit_shares{
	BYTEArrayVector(num_commits, commit_snd.cword_bytes),
	BYTEArrayVector(num_commits, commit_snd.cword_bytes)
};

BYTEArrayVector rec_commit_shares(num_commits, commit_rec.cword_bytes);
\end{lstlisting}
To fill these containers with the random commitments, the seed OTs for the \texttt{SplitCommit*} objects must be set. To set the sender's class, we must first construct a random number generator \texttt{osuCrypto::PRNG send\_rnd} and a communication socket \texttt{osuCrypto::Channel send\_channel.} 
\begin{lstlisting}
commit_snd.ComputeAndSetSeedOTs(send_rnd, send_channel);
\end{lstlisting}
The analogous operation must be applied to the \texttt{commit\_rec} object in another thread/program. Once seed OTs are set, the \texttt{SplitCommit*} objects can repeatedly be used to commit and decommit to many values. For example, the sender calls
\begin{lstlisting}     
commit_snd.Commit(send_commit_shares, send_channel);
...
commit_snd.Decommit(send_commit_shares, send_channel);
\end{lstlisting}
to interactively commits and decommits to \texttt{num\_commits} random values and stores the resulting commitment data in \texttt{send\_commi\allowbreak t\_shares}. It is important to note that these function calls results in two way communication across the  \texttt{send\_channel} and as such the receiver object \texttt{commit\_rec} must make the analogous function calls in a different thread/program. 

After the call to \texttt{Commit}, \texttt{send\_commit\_shares} holds two containers, where the $i$'th rows hold a 2 out of 2 XOR secret sharing of the committed codeword. The first \texttt{input\_bit\_size} bits of which is the actually committed value while the remaining bits contain the codeword parity information. For the case where $\texttt{input\_bit\_size} =128$, there is a utility function called \texttt{XOR\_128} for computing the random value that is committed to.
\begin{lstlisting}     
std::vector<uint8_t> value(input_bit_size / 8);
XOR_128(value.data(), send_commit_shares[0][i], send_commit_shares[1][i]);
\end{lstlisting}

Concurrently to the sender calling \texttt{Commit} and \texttt{Decommit}, the receiver must call 
\begin{lstlisting}     
if (commit_rec.Commit(rec_commit_shares, rec_rnd, rec_channel) == false) {
	throw std::runtime_error("bad commitment check, other party tried to cheat.");
}
...
BYTEArrayVector result_values(num_commits, commit_rec.msg_bytes);
if (commit_rec.Decommit(rec_commit_shares, result_values, rec_channel) == false)  {
	throw std::runtime_error("bad decommitment check, other party tried to cheat.");
}
\end{lstlisting}
Note that the receiver is required to check the return values of these function calls to ensure both the commitments and decommitments were correctly performed.

In addition to the \texttt{Decommit} function shown above, the library provides a more efficient mechanism for decommitting known as batch decommit. This method send roughly one fourth the amount of data over the network. The disadvantage of this approach is that it requires three rounds of interaction as opposed to a single round and performs additional computation. As already mentioned in \secref{usage_guidelines}, with some modifications to the library, these additional rounds could be eliminated using the Fiat-Shamir heuristic.

\subsection{XOR Homomorphic Operations}


Now we turn our attention to performing homomorphic operations on the committed values. Any subset of the committed values can be combined simply by XOR-ing their commitments together. For the common case of $\texttt{input\_bit\_size} =128$ the library provides the utility function \texttt{XOR\_CodeWords} to combine commitments. For example, the sender can create a new decommitment to the XOR of the $i$'th and $j$'th  commitment as follows
\begin{lstlisting}     
std::array<BYTEArrayVector, 2> new_decommit{
	BYTEArrayVector(1, send_commit_shares.cword_bytes),
	BYTEArrayVector(1, send_commit_shares.cword_bytes)
};

XOR_CodeWords(new_decommit[0][0], send_commit_shares[0][i], send_commit_shares[0][j]);
XOR_CodeWords(new_decommit[1][0], send_commit_shares[1][i], send_commit_shares[1][j]);
\end{lstlisting}
The receiver then must perform the analogous operation as follows
\begin{lstlisting}     
BYTEArrayVector new_decommit(1, commit_rec.cword_bytes);
XOR_CodeWords(new_decommit[0], rec_commit_shares[i], rec_commit_shares[j]);
\end{lstlisting}
Note that the sender must XOR together both of the shares that they hold while the receiver only performs a single XOR. These new commitments can then be decommitted as shown above.

\subsection{Bit Commitment}

For the case of $\texttt{input\_bit\_size} =1$, the user must manually compute the sender's committed value as 
\begin{lstlisting}     
uint8_t value = (send_commit_shares[0][i][0] ^ send_commit_shares[1][i][0]) & 1;
\end{lstlisting}
in lieu of the \texttt{XOR\_128} function. In addition, when performing homomorphic operations on the commitments, the function \texttt{XOR\_BitCodeWords} can be used. Finally, the values reported to the receiver in the \texttt{Decommit} and \texttt{BatchDecommit} function calls are packed. To obtain the value at a specific position, the utility function \texttt{GetBit} returns the value of the bit at a given index. Note that due to the bits being packed in the result container \texttt{result\_values}, it should be initialized as
\begin{lstlisting}     
 BYTEArrayVector result_values(BITS_TO_BYTES(num_commits), 1);
\end{lstlisting}

\subsection{Chosen Message}

To use this library with chosen message, additional work is required by the user. The central idea is that a commitment to a random value can be translated to a commitment of a chosen value simply by publishing the XOR difference between the two values, effectively forming a one-time-pad encryption under the random value which is committed. The user must maintain this ``translation" value for each commitment that it holds. To then perform homomorphic operations, the corresponding translation value must also be XOR-ed together, resulting in the translation value for resulting commitment. In particular, shown below is the operations for the sender to translate the 128 bit commitments made above to chosen message commitments. The value contained in \texttt{translation2} when XORed with the decommitted value of \texttt{new\_commit} will result in the XOR of \texttt{chosen\_message0} and \texttt{chosen\_message0}.
\begin{lstlisting}     
...

uint8_t * chosen_message0 = get_msg(0);
uint8_t * chosen_message1 = get_msg(1);

uint8_t translation0[16], translation1[16];

XOR_128(translation0,  commit_snd[1][0], commit_snd[0][0]);
XOR_128(translation0, translation0, chosen_message0);


XOR_128(translation1,  commit_snd[1][1], commit_snd[0][1]);
XOR_128(translation1, translation1, chosen_message1);

send_channel.send(translation0, 16);
send_channel.send(translation1, 16);

uint8_t translation2[16];
XOR_128(translation2, translation0, translation1);

...
commit_snd.Decommit(new_commit, send_channel);
\end{lstlisting}


	
	\bibliographystyle{alpha}
	\bibliography{bib}
	
	
\end{document}
