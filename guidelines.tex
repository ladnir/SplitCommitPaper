% -*- root:main.tex -*-

\subsection{Usage Guidelines}
\seclab{usage_guidelines}
It is a known fact that any UC-secure \cite{DBLP:conf/focs/Canetti01} commitment scheme in the standard model (non random oracle~\cite{DBLP:conf/ccs/BellareR93}) requires communicating at least the message size, in both the commitment and decommitment phase. The work of \cite{DBLP:conf/eurocrypt/GarayIKW14} was the first protocol to meet these bounds asymptotically for the non-homomorphic case, whereas \cite{DBLP:conf/tcc/FrederiksenJNT16} achieved the same for additively homomorphic commitments. It is an interesting observation, that although \cite{DBLP:conf/tcc/FrederiksenJNT16} does not achieve the above when decommitting to a single value, one can exploit the homomorphic property of the scheme to get around it when considering many or large messages. The technique is called batch opening and it works as follows for decommitting to $n$ messages:
\begin{enumerate}
  \item The sender sends $n$ messages directly to the receiver, which he claims are the decommitted values.
  \item The receiver then sends a challenge to the sender specifying $\ssec$ random linear combinations of all $n$ messages.
  \item The sender computes these $\ssec$ combinations, using the homomorphic property of the scheme, and decommits toward the receiver.
  \item The receiver verifies the decommits and in addition computes the same combinations on the $n$ initially claimed messages. If these match the correct decommitted combinations, he is guaranteed that the claimed messages are in fact what the sender originally committed to.
\end{enumerate}

We highlight that the above technique also works for a single large message, as the sender is this case can commit to the message block-wise to obtain the same effect.

\paragraph{Trade-offs and Considerations.}
We advocate that even applications where the \textit{homomorphic} property is not required, SplitCommit might be a right fit nonetheless. This is largely due to the concrete efficiency of the scheme as demonstrated in \secref{performance} and by the above batch decommit optimization. In addition, when committing to random messages, the scheme of \cite{DBLP:conf/tcc/FrederiksenJNT16} (and SplitCommit) allows for sub-linear communication in the commitment phase as these can be defined from the outputs of the PRGs directly. We here list the most profound pros and cons to consider before using the scheme:
\begin{description}
  \item[Pros:]
  ~
  \begin{enumerate}
     \item Concretely efficient, fraction of a microsecond to commit/decommit on standard hardware (see \secref{performance}).
     \item Additive homomorphism allowing for more flexibility and applicability.
     \item Possibility of sub-linear communication when committing to random messages.
     \item Standard-model assumptions, in particular no random oracle is required.
     \item If in a context already requiring OTs (such as general 2PC), the seed OTs required for the commitment scheme can be realized using OT extension, thus greatly minimizing the initial setup cost.
   \end{enumerate}
   \item[Cons:]
   ~
   \begin{enumerate}
     \item Two-party, directional commitments only. If the receiver needs to commit to messages as well, a separate instance of the scheme needs to be setup in the other direction.
     \item Requires OT to bootstrap. Depending on setting this can be too expensive to setup, especially when only a few commitments to small messages are required.
     \item Committing is interactive (single round).\footnote{For some settings, the Fiat-Shamir optimization~\cite{DBLP:conf/crypto/FiatS86} in the random oracle model can be applied to eliminate this interaction.}
     \item Non-interactive decommitment (non-batch) has $\sim4\times$ communication overhead for 128-bit messages with $\ssec=40$.\samefootnote
   \end{enumerate}
\end{description}



\subsection{Comparison to Random Oracle commitments}
It is widely know that a random oracle (RO) $h:\zo^* \to \zo^l$ can be used to implement a commitment scheme in a very straight-forward way. In order to commit to a message $m$, the sender computes $c = h(m\|r)$ for some fresh random string $r$ and sends $c$ to the receiver. To decommit, the sender reveals $m$ and $r$ to the receiver, who can then check whether $c = h(m\|r)$ or not. The above scheme can be proven secure in UC-framework if $h$ is modeled as an extractable, programmable random oracle.\footnote{Programmability is required to make the commitment equivocal.} With this assumption the simulator in the proof of security has complete control over the hash function and can both see what values are being hashed (extract) and decide on a particular output (program). This is clearly a very strong assumption on any real-world function (and impossible to achieve in general \cite{DBLP:conf/stoc/CanettiGH98}), and we note for completeness that several other commitment schemes exists using weaker flavors of ROs such as~\cite{DBLP:conf/tcc/HofheinzM04,DBLP:conf/ccs/Canetti0S14}. The latter schemes are however less efficient than the above sketched out scheme, and as such we take a worst-case approach by comparing SplitCommit to the most efficient RO based construction. Even so, it can be seen by the following discussions that the performance of SplitCommit is on par with, and sometimes supersedes that of the RO scheme on several interesting parameters.

\begin{description}
  \item[Assumption:] The RO scheme requires the existence of an extractable, programmable random oracle to be provably UC-secure, whereas SplitCommit requires the existence of UC OT. Although UC OT is also a strong starting requirement it is more generic than a random oracle.
  \item[Communication:] Typically the random oracle $h$ would be instantiated using a cryptographic hash function, say SHA-256 which then means the output size is 256 bits. Therefore no matter the size of the message (or if it's random or not), the sender needs to send 256 bits per committed value. This can be both a strength and a weakness, depending on the message size. Clearly if the message is large, the RO scheme can achieve sub-linear communication when committing, even for chosen messages. However for smaller messages, such as \eg commitments to 128-bit strings, committing to a value has a 2x overhead. For decommitting, the sender needs to send the message and the random string $r$ which is typically around 128 bits long. Again for the case of 128-bit messages this yields a 2x overhead.

  For SplitCommit the cost of committing can be sublinear for random messages and essentially the message size for large chosen messages. The sublinear communication is only true for somewhat large messages and the exact cutting point depends on the security parameter and ECC parameters, but as an example, for 128-bit messages and $\ssec=40$ it requires 134 bits to commit to a random message and 128-bits to decommit using batch decommit.

  \item[Computation:] From the performance section we see that SplitCommit can commit and decommit to a value in less than a microsecond (when considering many values). Based on the observations of \cite{DBLP:conf/pkc/CascudoDDGNT15} this roughly equivalent to a SHA-256 evaluation.
  \item[Homomorphism:] SplitCommit allows additively homomorphic decommits whereas the RO scheme does not.
  \item[Setup Cost:] SplitCommit requires an intial setup phase involving a number of potentially expensive seed OTs. However, depending on the scope and complexity of the application this can be considered anything from a complete deal-breaker to a negligible added cost. The RO scheme requires no such setup.
  \item[Interactiveness:] The RO scheme is non-interactive in both the commitment and decommitment phase, whereas SplitCommit in it's most efficient mode is interactive in both phases. However as previously mentioned, using the Fiat-Shamir heuristic (which as well requires a RO assumption) this can be turned non-interactive as well.
\end{description}


\begin{figure}
	\small
	\begin{tabular}{|c||c|c|c|c|c|c|c|}
		\hline
		Scheme                 & Setup & Homomorphic & Interactive &\shortstack{Random \\Commit} &  Commit & Decommit & \shortstack{Batch \\Decommit} \\ \hline
		\cite{DBLP:conf/tcc/FrederiksenJNT16} &  $\kappa$ OTs  &     \textbf{Yes}     &     Yes     &   \textbf{134 bits}    &   262 bits    & 524 bits &    \textbf{128 bits}    \\ \hline
		Random Oracle             &      \textbf{ No}       &     No      &     \textbf{No}      &   256 bits    &   \textbf{256 bits}    & \textbf{256 bits} &    256 bits    \\\hline
	\end{tabular}
	\caption{Comparison to the random oracle commitment scheme.}
\end{figure}