% -*- root:main.tex -*-

\section{Performance}
\seclab{performance}
To demonstrate the scalability of the implementation, we perform benchmarks on an Intel Xeon server consisting of 2x 36-core Intel Xeon CPUs (E5-2699 v3 @ 2.30GHz) and 256GB of RAM. In the single threaded setting on a 10Gbps LAN, $n=2^{24}$ commitments on random 128 bit values requires 11 seconds, resulting in 0.65 microseconds per commitment. When decommitting via sending the two secret shares for all $n$ commitments, the implementation requires 5.4 seconds, 0.32 microseconds per decommitment. The same operation performed using batch decommit requires 4.6 seconds, 0.27 microseconds per decommitment. Furthermore, homomorphic operations can be performed extremely fast, simply requiring  a few XOR instructions per operation, requiring less than 1 second for $n$ operations.

We also benchmark our implementation in the 100Mbps WAN setting with a 80ms round trip latency. For $n=2^{24}$, committing  requires 39 seconds, 2.3 microseconds per commitment. Decommiting via sending both secret shares required 106 seconds, 6.2 microseconds per decommit, while the more communication efficient batch decommitment method required 30 seconds, 1.8 microseconds per decommitment.

\begin{figure}
\begin{center}
  \small
  \begin{tabular}{|c|c|c|c|c|}
  	\hline
  	      bit size       &   Network   & Commit & Decommit & \shortstack{Batch \\Decommit} \\ \hline
  	\multirow{3}{*}{128} & 10Gbps LAN  &  0.65  &   0.32   &             0.27              \\
  	                     &  1Gbps LAN  &  1.2   &   2.0    &             0.79              \\
  	                     & 100Mbps WAN &  2.3   &   6.2    &              1.8              \\ \hline
  	 \multirow{3}{*}{1}  & 10Gbps LAN  &  0.07  &   0.12   &             0.13              \\
  	                     &  1Gbps LAN  &  0.22  &   0.56   &             0.57              \\
  	                     & 100Mbps WAN &  0.56  &   1.5    &              1.6              \\ \hline
  \end{tabular}
  \caption{Amortized running times in microseconds ($\mu s$) for $n = 2^{24}$ commitments.}
  \end{center}
\end{figure}

When performing bit commitments in the LAN setting, $n=2^{24}$ commitments to random bits requires 1.3 seconds and decommitting requires 2 seconds. Interestingly, batch decommitting is slightly slower, requiring 2.3 seconds to decommit. We attribute this to the reduction in communication being outweighed by the added computation that is required by batch decommit. Indeed, in the 100Mbps WAN setting where communication is more expensive, batch decommitting is faster, requiring 60 seconds as opposed to 61.

In terms of communication overhead, committing to a random 128 bit value requires 134 bits, with an added 128 overhead for chosen message. Decommitting requires 524 bits, while batch decommitting only requires 128 bits with a constant additive overhead of 20,960 bits, independent of the number of decommitments. Committing to bits requires 40 bits of communication and decommitting requires 80 bits. Batch decommitting only requires a single bit of communication with a constant additive overhead of 3,200 bits, again independent of the number of decommitments considered.

